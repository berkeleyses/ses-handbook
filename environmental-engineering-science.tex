\chapter*{Environmental Engineering Science}

\textbf{Introduction}: The Environmental Engineering Science (EES) major is a rigorous interdisciplinary program pairing engineering fundamentals with courses in the environmental and natural sciences. Although environmental engineering options may be found in the chemical, civil, mechanical, and material science engineering departments, the EES curriculum provides a broader foundation in the sciences. At the same time, it allows students to focus their study on environmental issues more than any other program option in the College of Engineering. As a student in this program, you will have many chances to direct your own curriculum. Thus, it is important that you plan ahead and research course prerequisites, units, and availability.

\textbf{EES vs. CEE with Environmental Emphasis}: Many lower division students with a strong interest in environmental engineering have trouble choosing between the EES track and the Civil Engineering with an environmental emphasis.

\begin{itemize}
  \item For starters, your diploma will say B.S. Environmental Engineering Science or B.S. Civil and Environmental Engineering, depending on your decision.
  \item Civil Engineering requires a few more courses not environmental in nature (to be expected), including Civil Engineering Materials (CE 60), Mechanics of Structures (CE 130), and Engineering Data Analysis (CE 93). A Civil Engineering student is also required to take both a design elective as well as four core courses, not all of which are related to environmental engineering.
  \item In contrast, the EES program requires more breadth and applications to the earth sciences, architecture, or chemistry (depending on your choices; see Advanced Science Sequence section). For example, EPS 108 (Field Geology; includes field trips every week), EPS 50 (The Planet Earth), and Architecture 140 (Energy and the Environment) will count toward your major. Thermodynamics, hydrology, math, stats, and computing are also required in EES.
  \item Both majors require a number of upper division technical electives (EES with 12 units, CE with 15 units).
  \begin{itemize}
    \item CE requires that these electives exist within the College of Engineering or Chemical Engineering, while EES electives may encompass technical courses in or out of engineering.
    \item CE electives may be any combination of classes (so long as they are engineering-related), while EES has what are known as Clusters. Essentially, the electives must fall within a specific subject area, or Cluster. For each Cluster (Air Pollution and Climate Change, Biotechnology, Ecosystems and Ecological Engineering, Environmental Fluid Mechanics, Geoengineering, Water Quality), one should examine the specific lists of allowed classes. (Do note that these clusters can be flexible with adviser approval.) Look out for classes with prerequisites (e.g., in Electrical Engineering, Mechanical Engineering, MCB) not often taken by EES students. 
  \end{itemize}
  \item Unlike the Civil Engineering major, the EES program does not yield an ABET-accredited engineering degree. However, the flexibility of the EES program allows each student to specialize in what they are most interested in, while a Civil Engineering degree covers the breadth of civil engineering practice with some emphasis on environmental engineering.
  \item Visit the Engineering Guide for the full requirements:
  \begin{itemize}
    \item {\fontfamily{qcr}\selectfont engineering.berkeley.edu/academics/undergraduate-guide}
  \end{itemize}
\end{itemize}

\textbf{Lower Division Basic Science Electives}: One key decision occurs during your sophomore year when you can choose three basic science electives from this list: Physics 7C, Bio 1A/1AL, 1B, Chem 1B, 3A, 3B, and EPS 50. Those who have taken AP, IB, or A-Level exams in Biology should check to see if they can pass out of Bio 1A/1AL and Bio 1B (link at {\fontfamily{qcr}\selectfont admission.universityofcalifornia.edu/counselors/exam-credit}). For those possibly considering grad school, many programs, such as in the earth sciences, recommend or require one full year of chemistry, so Chem 1B may be a wise choice. The Chemistry 3 series is relevant to many aspects of environmental engineering, such as toxicology, biofuels, and soil chemistry. EPS 50 is helpful for the Geoengineering cluster. Those more physics-minded (for example, those interested in the Atmospheric Science Advanced Science Sequence) might find Physics 7C useful.

\textbf{Clusters}: Build cluster/sequences carefully. Make sure you have the necessary knowledge to get the job you want. Explore different possibilities. Remember that you can’t knock out series and cluster course requirements with one class. Not all classes are offered every semester! Also, there are many classes not listed in the requirements that you can get approved into a cluster or sequence! Consult your faculty adviser for more information.

\textbf{Advanced Science Sequence}: In accordance with the program’s goal of a broad science background, students are required to take at least 8 units in one of the following sequences: Atmospheric/Climate Science, Biochemistry/Microbiology, Ecosystems/Soils, Geology/Geodynamics, Organic Chemistry, or Physical Chemistry. For more information, consult the EES requirements.

\textbf{Fluid Mechanics Elective}: CE 100 is the most commonly taken fluid mechanics course by EES students. Unlike ME 106 and ChemE 150A (the other two options), CE 100 discusses open channel flow applicable to streams and rivers, an important aspect of environmental engineering. In addition, those considering the other two classes should look up and consider the respective prerequisites.

\textbf{Thermodynamics Elective}: ME 40 is the standard and likely the easiest option. Those with an eye toward chemistry should certainly consider ChemE 141 and E 115, the other two choices. Note that the Civil Engineering graduate program likes to see E 115 completed (since it is upper division).

\textbf{Hydrology Elective}: CE 103 covers the hydrological cycle, floods, runoff analysis, and watershed modeling, while CE 115 involves a significant amount of chemistry and covers topics such as pH, alkalinity, acid/base speciation, metal chemistry in aqueous systems, and redox chemistry.

\textbf{Upper Division Math/Stats/Computing Elective}: Lastly, you will be required to select an advanced mathematics course in your junior year. Perhaps the most practical are the two statistics options. In particular, Stat 133 (Computational Statistics) is very practical and teaches R (a programming language), but the class is in high demand. Some, including Math 128A and E 177, emphasize programming (using MATLAB). Others are pure math classes, including Math 104, 110, 126, and 185. Students are advised to read further on individual class descriptions to determine which is best for them.

\textbf{Other Recommendations}: Many EES students highly recommend Professor Nazaroff’s CE 111 (Environmental Engineering) and CE 107 (Climate Change Mitigation). Taking a class that has a big semester project (group or individual) can be a great experience (and really boost your rsum). Try classes outside your major or go for a minor (or double major)!

\textbf{Minors and Double Majors}: It can be fairly easy to minor in a related field or double major. In particular, look at Architecture, Earth \& Planetary Science (EPS), Forestry, Molecular Environmental Biology (MEB), Sustainable Environmental Design (SED), and the College of Environmental Design (CED). You might find a graduate program is right for you in an outside department! For example, some EES graduates are completing a master’s program in EPS.

\textbf{Last Words}: Talk to professors, but don’t limit yourself to talking to professors just in the major. There are people in Architecture, EPS, Forestry, MEB, and Sustainable Design EVERYWHERE that can have interests similar to you and have opportunities for you. The Engineering Student Services (ESS) and departmental advisers are really friendly and accessible. Be sure to consult frequently with your faculty adviser, departmental adviser, AND the ESS office in 230 Bechtel for detailed information about courses/degree requirements.

