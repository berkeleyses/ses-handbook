\chapter*{Engineering Physics}
\addcontentsline{toc}{chapter}{Engineering Physics}

“Scientists investigate that which already is; Engineers create that which has never been” - Albert Einstein. Umm, why not do both!?
 
As an Engineering Physics (EP) major, you can pick from a range of foci – an engineering mindset with a strong understanding of physics. You might be interested in semiconductors, but you will probably know more solid-state physics than most electrical engineers. You might be interested in building machines, but you will probably know more advanced methods in mechanics than most mechanical engineers. Engineering Physics starts you off with more of a first-principles approach! If you’re curious about the foundations of technology, these extra physics classes help quench that thirst.
 
Chief among the attractions of Engineering Physics is the ability to take diverse classes. Go wild! Still, it helps to emphasize a particular area of study. Choose your curriculum according to your interests.
 
\textbf{Lower Division:} Many students use the AP credits to pass out of Math 1A/1B. Once you finished Math 1A/1B/53, you can choose between Math 54 and Physics 89. Both courses teach you the skills needed for upper-division courses, but Physics 89 is more physics-oriented because the applications to physics are emphasized. Math 54 is more general and gives you a better mathematical intuition of linear algebra and differential equations. For chemistry, many students use the AP credits to pass out, but if you are particularly interested in chemistry or chemical engineering, consider taking Chem 4A even you already got credit for Chem 1A. This class is required for students in the College of Chemistry and gives you are more solid understanding of chemistry principles.
 
For the programming requirement, if you want to gain general coding experience, we would recommend CS 61A. In this class, you will learn Python, Scheme, and develop good coding practices. If you want to gain coding experience for engineering, we would recommend Engineering 7. This class is required for most engineering majors such as Mechanical Engineering, and you will learn MATLAB with high-performance numerical computation and visualization. The last option is Physics 77. This class is designed for Physics students in L\&S who want to gain some relative coding experiences; it is less challenging than CS 61A and Engineering 7. But if you decide to take Physics 77, we would recommend you also to take Physics 88, a 2-unit useful class that teaches you data science with applications to physics.
 
For the lower division electives, you need to take three courses from the list: Astro 7A, Astro 7B, Chem 1B or Chem 4B, Chem 3A/3AL, CS 70, MSE 45 (45L are recommended), EE 16A, EE16B, Bio 1A/1AL, Bio 1B.
 
Here are some of the path students took according to their interests:
\begin{itemize}
    \item Astrophysics: Astro 7A + Astro 7B + One Lower-division technical elective
    \item Biology: Bio 1A + Bio 1B + One Lower-division technical elective
    \item Chemistry: Chem 4B, Chem 3A/3AL + One Lower-division technical elective
    \item Electrical Engineering: EE 16A, EE 16B, CS 70
    \item  Mechanical Engineering: ME C85, EE 16A, EE 16B
    \item Material Engineering: MSE 45/45L, EE 16A, EE 16B
\end{itemize}
Many of the courses listed above are important prerequisites for upper division classes, and some are only offered in either fall or spring.  Check the Berkeley Academic guide and be sure to plan ahead!

For Physics requirements, we strongly encourage taking the Physics 5 series. The Physics 5 series is designed for Physics majors, and the added rigor and challenge will help you adjust to upper-division physics and mathematics courses more quickly. Most Physics majors take the 5 series, so that is where you can meet the students that will be in your upper-division physics classes. The specialized lab courses for Physics 5 series help you to build a stronger understanding of the experimental side of physics, which will help you to succeed in the upper-division lab classes. On the other hand, Physics 7 series is less challenging, so if you have a heavy course plan, Physics 7 series can also be a good option.
 
\textbf{Upper Division:} You will notice that your upper division requirements allow you to choose either an engineering course or the equivalent physics course for a broad range of subjects. If your goal is physics graduate school, it is recommended that you take Physics 105 (Mechanics), 112 (Thermodynamics), 137A/B (Quantum Mechanics), and 110A/B (Electrodynamics) in order to be prepared for graduate studies.  This isn’t a strict requirement, but all of these topics are incredibly important for the Physics GRE, a test that’s critical for grad school applications. On the other hand, if your goal is to enter a specific industry (say, for instance, robotics), then choose engineering courses related to your future career path. Or, you can take both!
 
For the lab courses, Physics 111A is most useful for basic engineering and physics research – it equips you to understand experimental technique and instrumentation, which is ubiquitous.  Physics professors highly value this course on your resume if you are pursuing undergraduate research. EE 143 is also very useful (and is recommended by Professor Attwood), especially if you are interested in semiconductors and fabrication techniques.  The specific skills developed in this class can make you highly desirable to employers in these fields, as well as open up opportunities for research in the electrical engineering department. NE 104 is designed for students who are interested in nuclear radiation and helps students build a stronger understanding of the application of atomic physics. All of these classes can help you to be a great undergraduate researcher, so don’t hesitate to take them earlier on if you think it might interest you.
 
For math, you have a choice between the 121 series and 104/185. In general, students are split on which ones to take. We’ll try and distill the pros and cons here. The 104/185 sequence can be difficult and time-consuming; however, they more adequately prepare you for graduate school in theoretical physics. Many students appreciate the logical clarity of Math 104 and that it demystifies “rigorous” math. Math 185 (Complex Analysis) provides you a strong mathematical background, teaching theorems that underpin many core principles of quantum mechanics, E\&M, and signal processing.  This is a difficult class, but it will pay off if you intend to go more into theory.  If you want to know how to apply advanced math in engineering and physics, that is where the 121 series comes in. It prepares you for math you will see in upper-division physics courses (special functions, solving different differential equations, and Fourier series). The 121 series is about as time-consuming as 104/185, but sacrifices some curricular cohesion to avoid excessive rigor.  The 121 series is sometimes described as a jumbled math toolbox, whereas 104/185 is sometimes attributed as a series where you struggle to prove obvious facts like y = x has a slope of 1.  If you are struggling to decide, ask your friends that have already taken them!  Engineering physics majors tend to have strong opinions on this topic.
 
Generally, the engineering courses tend to hone your numerical-spatial problem-solving skills, while physics courses will demand critical thinking with more equations and almost no numbers.
Here are some key notes for upper-division classes:
\begin{itemize}
    \item ME 104 and ME 106 have MEC85 and Engineering 7 as prerequisites.
    \item EE 117 and EE 118 have EE 16A and EE16B as prerequisites.
    \item Physics 141A has Physics 137A/B as prerequisites.
    \item Engineering 40 counts as an upper-division class for the thermodynamics requirement.
    \item NE 104 has NE 101 as a prerequisite.
    \item Students planning to pursue graduate school in physics are advised to complete Physics 111B (for 3 units) to satisfy the laboratory requirement.
    \item The 15 units of upper division engineering and 14 units of upper-division physics include the ones directly listed as requirements.  Essentially this means you have to balance your physics and engineering workloads.
    \item At least 40 units of approved upper division technical subjects (mathematics, statistics, science, and engineering). These 40 units DO include all required upper division technical course work taken for the major.
\end{itemize}

Have questions or need help for classes? Ask the major representative from SES or the faculty advisor! Remember to consult with your faculty adviser often, and talk about potential plans for future semesters. Your advisers are on your side – make sure to take advantage of the guidance they can provide!


