\chapter*{Engineering Physics}

As an Engineering Physics (EP) major, you can pick from a range of focuses – an engineering mindset with a fundamental background. You might be interested in semiconductors, but you would study more solid state physics than most electrical engineers. Or, you want to build things, but still work though Lagrangian and Hamiltonian methods of classical mechanics. Engineering Physics starts you off with more of a first-principles approach! If you’re curious about the foundations of technology, these extra physics classes help quench that thirst.

Chief among the attractions of Engineering Physics is the ability to take diverse classes. Go wild! Still, it helps to emphasize a particular area of study. Choose your curriculum according to your interests at the time. If you keep thinking about how materials work, take E 45. If you are finding an interest in theory, try programming – a useful tool to run simulations. (Lecturer Reinsch in the physics department advocates that all physics undergraduates learn to program!)

Read the fine print in the General Catalog ({\fontfamily{qcr}\selectfont guide.berkeley.edu}) and know what the prerequisites are for a class. That way, you can take your requirements before taking upper division classes. Since some classes are offered in only the Fall or Spring semester, missing a prerequisite can become a hassle. Key survey classes that are important prerequisites for many interesting upper division classes are E 45, EE 16A/B, and ME C85/CEE C30.

Some of the required classes have a list of recommended – but not strictly required – classes. Ask the professor if the prerequisite is actually necessary. Many students agree that you will be best off if you use Advanced Placement credit for any math and chemistry classes you can. This should give you more time for the really interesting and important upper division classes. It could also help make your semester unit loads more bearable. Use the time to develop relationships with your peers in physics – many will take a few upper division classes with you. We strongly encourage taking the Physics Honors (H7) sequence. The added rigor and challenge will help you adjust to upper division physics and mathematics courses more easily.
In your first year, it is fairly standard to focus on lower division physics, math, chemistry, and miscellaneous breadths. You can make important decisions (such as what topic to focus on) in your sophomore year, so don't stress out just yet! If you do have open slots, be sure to explore intro classes, such as Astro 7A, CS 61A, EE 16A/B, or Math 55. These classes can help you decide your interests so you can plan out your courses over the next four years.

\textbf{Depending on how you use your electives, this major is well-suited for graduate studies in physics, engineering, and materials science. Graduates regularly go to the best graduate departments in all of these areas.} You will notice that your degree requirements allow you to choose either an engineering course or the equivalent physics course for a broad range of subjects. If your goal is physics graduate school, it is recommended that you take Physics 105 (Mechanics), 112 (Thermodynamics), 137A/B (Quantum Mechanics), and 110A/B (Electrodynamics) in order to be prepared for graduate studies. On the other hand, if your goal is to enter a specific industry (say, for instance, aerospace engineering), then choose engineering courses related to your future career path. \textbf{Working in a research group and at an internship are both enlightening experiences; if you can, try both.}

From the lab courses, Physics 111 is the most useful for basic engineering and physics research – it equips you to understand experimental technique and instrumentation, which is ubiquitous. EE 143 is also very useful (and is recommended by Professor Attwood), especially if you are interested in semiconductors and fabrication techniques. Both of these classes can help to make you a great undergraduate researcher.

For math electives, you have a choice between the 121 series and 104/185. Unless you are really interested in theory and abstract math, take the 121 series! The 104/185 sequence can be difficult and time-consuming, which may impact your understanding and performance in other classes. Alternatively, the 121 series prepares you for math you’ll see in upper division physics courses (special functions, solving different differential equations, and Fourier math). Also, more of your peers will be in the 121 series, so you’ll have a good study group and a more familiar curve. The 104/185 courses, however, more adequately 
prepare an individual for graduate school in a theoretical field. In particular, Math 185 (Complex Analysis) is applied in both quantum mechanics and signal processing. Additionally, if you plan on taking many upper division physics courses, consider taking Physics 89, a mathematics course targeted for physics majors.

After you finish your lower division courses, there are a lot of opinions about which physics upper division courses you should take first. Some people say Physics 105 (Mechanics) is quite fundamental, but it’s so time-consuming you should save it for the end. Physics 112 (Thermodynamics) can be easy or hard depending on your instructor, but little thermodynamics topics do show up in other classes like quantum mechanics and solid state physics. It’s not necessary, but it is nice to have! You’ll find that every course in upper division physics is slightly related to all of the other upper division physics courses.

Because you will have to make choices between engineering and physics courses, it may be helpful to have student descriptions of classes in addition to the General Catalog. (Don’t know anyone who’s taken that class? Ask SES!) Remember that the engineering courses tend to hone your numerical-spatial problem-solving skills, while physics courses will demand critical thinking with more equations and almost no numbers.

\textbf{Remember to consult with your faculty adviser often, and talk about potential plans for graduate studies. Your advisers are on your side – make sure to take advantage of the guidance they can provide!}

