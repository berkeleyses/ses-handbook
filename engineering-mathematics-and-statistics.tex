\chapter*{Engineering Mathematics \& Statistics}
\addcontentsline{toc}{chapter}{Engineering Mathematics \& Statistics}

This major is a combination of math and engineering. Due to the breadth of this program, it is especially suited for those who plan to enter graduate school. If you are unsure of engineering but still love math and science, Engineering Mathematics \& Statistics (EMS) is still for you. Additional engineering or science deficiencies for graduate work that you may have can be made up early in graduate school, and your broad background can offer some creative sparks for research. If it turns out that engineering is not right for you, you still will have important knowledge of it that can help you in other fields. This is an excellent program for additional math coursework and a stepping-stone to graduate work in math or science education. Alternatively, you may develop a pre-actuarial plan with technical courses in math, statistics, operations research, and engineering economics, and with social science courses in economics, business, and law.

It is important to emphasize that lower division math classes are very different from upper division classes. Upper division classes are more theoretical and have little emphasis on computation. For additional guidance, you may want to contact the mathematics department. The department office is located in 970 Evans. Also, the Mathematics Undergraduate Student Association (MUSA) can be very helpful. They may be contacted by e-mailing {\fontfamily{qcr}\selectfont musa@math.berkeley.edu}.

You can find many potential clusters of engineering courses that will give you some concentration. One way is to look at the courses required or recommended for other engineering majors and their various options. The College of Engineering Undergraduate Guide ({\fontfamily{qcr}\selectfont coe.berkeley.edu/guide}) is a good source to look at classes required for other majors. Of all the engineering departments, the Industrial Engineering and Operations Research (IEOR) department is closest to EMS. Look at the IEOR 160 series.

Your upper division math/stat classes require Math 128A (a class in numerical analysis for which prior exposure to a high level programming language is recommended), Math 110 (linear algebra, a course very similar to Math 54), Stat 134 or IEOR 172, 3 electives in math or 
statistics from an approved list, and, most importantly, one of the following two-course sequences:

\begin{enumerate}
  \item Math 104, Introduction to Analysis, and Math 105, Second Course in Analysis.
  \item Math 104 and Math 185, Introduction to Complex Analysis.
\end{enumerate}

Note that Math 104 is a rite of passage where you basically prove calculus.
However, be aware that it is NOT an easy course.
To quote the Math Department’s Undergraduate Announcement, “students are advised that this [Math 104] is a very difficult course, and is best taken following another upper division mathematics course.”
Typically, Math 110 and Math 128A are considered to be easier upper division courses.

Broadly speaking, Math 105 covers further topics in analysis and gives an introduction to the beautiful subject of integration theory, which is taught in the introductory graduate analysis course.
On the other hand, in Math 185 you will see the application of analysis techniques to functions of complex variables.
Complex analysis finds many engineering applications, ranging from signal processing to fluid mechanics and quantum mechanics.
If you are interested in these applications, you should take Math 185.
If you plan on pursuing graduate studies in mathematics, you should take both!

If you are more interested in applied mathematics more than pure math, then the math electives that may be of interest are:

\begin{itemize}
  \item 123 Ordinary Differential Equations
  \item 126 Partial Differential Equations
  \item 128B 2nd Course in Numerical Analysis (may not count as an upper-division technical elective)
  \item 170 Mathematical Methods for Optimization
\end{itemize}

The four basic core courses for the math major in L\&S are Math 104, 185, 110, and 113 (Abstract Algebra).
Thus, if you are really interested in math, you should choose 113 as an elective. L\&S math majors also need to take one course from each of two of the following subject areas: Computing (which you automatically satisfy because of Math 128A); Geometry (Math 140, 141, 142); Logic and Foundations (Math 125A, 135). These courses are possibilities for the three-course elective. If you are interested in statistics, then you should consider completing the full Stat 134-135 sequence to have a more complete background in Probability (Stat 134) and Statistics (Stat 135), even though there are statistics courses satisfying the elective requirement which require only 134. Just for reference, L\&S statistics majors are recommended to take Math 104, 105, 113, 126, 128A, and 185. If you can, give the honors sections a try when you take Math 104, 185, 110, and 113, especially 104. These are rigorous and difficult sections, but they are worth the work if you really love and appreciate the beauty of mathematics.

Although it is not required, you may want to take Math 55 (or CS 70) on discrete mathematics. This class introduces proof methods and techniques, and it helps you adjust to upper division math classes, which tend to be far more theoretical than lower division classes. Discrete mathematics also places a lot of emphasis on structures (such as sets, graphs, relations) and less on computation, so it exposes students to sides of math they haven't seen before. You get a chance to write clear, concise proofs, and get a “flavor” of upper division mathematics. (This counts as a lower-division elective!)

Within computer science, the 150 series are computer architecture courses; the 160 series, software; the 170 series, computer theory; and the 180 series, computer applications. 
Also, check out the 120 series in the Electrical Engineering department, which includes signal processing, communication, and random processes, which are all subjects which go hand in hand with the mathematical background and engineering focus of the major.

The requirement for a computer science minor is CS 61ABC, CS 70, and three upper division courses in computer science.
Unfortunately, only one upper division CS course can overlap for the CS minor, but at least you can have the equivalent of a CS minor, which is still something for the resume. Another feasible minor is chemical engineering provided completion of either the Chem 1 or Chem 4 series.

If you are interested in the physics side of applied mathematics, you may also want to consider concentrating on physics courses such as fluid dynamics or quantum mechanics. Of course, talk with your faculty adviser when planning your technical electives.  Also consider talking to some of your Engineering Science neighbors over in Engineering Physics.

Because any two EMS students can potentially have drastically different programs, it is very important that you glean as much as you can out of your faculty adviser and talk to people in the many fields in which you are interested. Ask your faculty adviser and your professors for people that can give you more information. If they refer you to another professor, do not be afraid to go to his/her office hours even though you are not in his/her class.

