\chapter*{Career \& Internships}
\addcontentsline{toc}{chapter}{Career \& Internships}

\section*{How to Get a Research Job}

If you are interested in doing research within an engineering science group, you should talk to faculty advisers and talk to the professors directly. If you would like to work in another department, such as Mechanical Engineering, EECS, Molecular Cell Biology, or Chemical Engineering, obtain the departmental announcement which contains a detailed listing of professors and what they are currently working on. From this list you can call or email the professor to make an appointment. Prepare yourself when meeting the professor. Look professional and act interested. Ask questions to demonstrate that you have some knowledge of the professor’s research. Have a r\'{e}sum\'{e} and a copy of your transcript ready. Consider beforehand whether or not you want to be paid (very often lab budgets don’t initially allow for this) and how much time you want to put in. If the first interview doesn’t work out, don’t be discouraged. Just try again! The most important part of the process is starting. Once you start and get some contacts, eventually you will get a research job. Also, try looking at URO (Undergraduate Research Opportunities Program) through the College of Engineering or URAP (Undergraduate Research Apprenticeship Program) through the College of Letters and Sciences.

\section*{If You’re Not Going to Graduate School}

The majors in Engineering Science are geared towards graduate/medical studies. Thus, most of the curriculum focuses on the theoretical aspects of the various disciplines, leaving out some of the more practical engineering courses. This may put you at a disadvantage when competing for an industrial position against a mechanical or chemical engineer, who has a stronger practical-industrial background. On the other hand, holders of a Ph.D. degree usually take research and development positions. Nevertheless, there is no reason to despair. If your purpose and desire is to get a position in industry (and start earning good money in the process) with just a bachelor’s degree, then these humble pointers might be of some help:

\begin{enumerate}
  \item Decide your field; try to take as many related engineering courses as possible. This might seem something painful to do since this may imply giving up your “easy” electives for some “hardcore” engineering ones, but the payoff will be landing a good job once your weaknesses have become your strengths.
  \item Get an engineering-related job before graduating. This is a must since this is what employers look for when they hire someone. Anything will work, i.e., part-time, full-time, paid, or volunteer.
  \item Remember that experience is what counts. For job listings and references, go to the Career and Graduate School Services Center located at 2440 Bancroft Way. If it is too hard for you to work during the semester, try to get a summer job or take a semester off and get a CO-OP. Taking a CO-OP or an internship is the best way to get a permanent job. Always have your rsum ready. This is crucial. “Don’t
leave home without it.” The handbook available at the Career Center has very valuable rsum advice and examples. You may even want to enroll in one of the many rsum -writing classes offered on campus.
\end{enumerate}

\textbf{Engineering Science programs are not ABET accredited, so consider taking the Fundamentals of Engineering (FE) exam before graduation!}

\section*{Internships}

An internship is a wonderful and effective way to connect your academic experience with the professional work arena. It allows you to gain valuable exposure to the workplace, provides the opportunity for skill development, and gives you a competitive edge in the job search.

\section*{Getting Started}

Visit {\fontfamily{qcr}\selectfont career.berkeley.edu/Internships/Internships.stm}.
\begin{itemize}
  \item Internship Listings
  \item Frequently Asked Questions
  \item Meet with a Counselor
  \item Externships – Short term “job shadowing” opportunities
  \item Campus Opportunities – Campus-affiliated internships and experiential opportunities
\end{itemize}

