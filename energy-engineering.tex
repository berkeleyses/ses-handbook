\chapter*{Energy Engineering}

Energy Engineering (EnE) offers an interdisciplinary look at the challenges around the quickly-changing energy landscape. Energy Engineering majors take the same strong quantitative base that other Berkeley engineers take, including physics, linear algebra, single and multivariable calculus, as well as chemistry. It is currently a small major, but every year since its creation in 2012, the major has increased rapidly, some years even doubling in size.

In addition to these core quantitative classes, a huge variety of other perspectives shed light on the energy issue. Required classes come from departments all over the College of Engineering, as well as outside it. The required Civil Engineering classes generally take a bird’s-eye view, focusing on fluid and energy flows at large scales, as well as the associated chemical reactions. From Mechanical Engineering, EnE pulls in a more rigorous mathematical approach to the physical and energetic interactions between and within systems. Electrical Engineering classes seem to split the difference, with EE 137 (Electric Power Systems) taking a high-level view of power systems and EE 134 (Photovoltaics) taking a detail-oriented view of semiconductors. Statistics, sustainability and economics courses also inform the study of energy. Each of these categories pulls a class from a set of possible classes listed on the degree worksheet. For example Stat 134, IEOR 172, or Math 55 can be used for the statistics requirement. Each brings a different perspective to the intersection of statistics and energy. All of these skills come together in a senior capstone project E 194, which allows students to synthesize and apply the knowledge they have gained to a current problem in energy engineering.
 
Due to the number of classes, an Energy Engineering program is a bit less flexible than other Engineering Science majors, but you can be assured that you are taking basically every energy-related class that Berkeley offers. This rigidity requires a bit of careful planning to make sure that you have all of the prerequisites at the right time. Certain classes are offered only fall or spring, so this is another consideration. The schedule is actually more flexible than it may seem at first because of the major’s relative novelty. Energy Engineering is still in flux, so if 
you see a class that piques your interest, speak up! You may be able to substitute it for one that feels less relevant. Faculty and departmental advisors can give helpful advice and will be important voices in developing your path through and after university. With this ability to tailor the major comes the responsibility of looking ahead and doing so thoughtfully.
 
Student groups are very important in determining what you want to do with your major, as well as for meeting people with similar interests. (For example, try SES!) BERCU attracts a variety of majors and puts on some really cool events, occasionally fueled by free pizza. Engineering project clubs such as Human Powered Vehicle and CalSol are also great ways to hone skills and meet people.
 
EnE’s interdisciplinary nature allows the pursuit of a variety of different career paths including graduate school, mechanical engineering, solar photovoltaics at either the system level or the semiconductor level, and more. The Career Center sends out newsletters about research or internship opportunities, and signing up for BOTH engineering and environmental newsletters will allow you to see a large range of possibilities. (For more information, see https://career.berkeley.edu/MailList/MailList.) Because there are no “Energy Engineering” career fairs (yet!), you tend to make your own way by attending a variety of career events.
 
Examples of relevant challenges include improving energy efficiency in buildings, transportation, and industrial sectors; reducing emissions of carbon dioxide and other environmental impacts associated with fuel combustion in power plants and engines; designing and managing smart energy grids; developing non-fossil energy systems including wind, solar, geothermal, and nuclear power; and developing and improving production, storage, and distribution infrastructure needed for electric vehicles, hydrogen fuel cells, and biofuels.

\textbf{Final Note}: Many EnEs recommend taking Thermodynamics early (to account for scheduling), and to avoid taking both CE 107 and ERG 100 (the content is quite similar). Also, take anything Nazaroff teaches! For a list of FAQs about the major and other information, visit http://engineeringscience.berkeley.edu/energy-engineering.

