\chapter*{Energy Engineering}
\addcontentsline{toc}{chapter}{Energy Engineering}

Energy Engineering (EnE) offers an interdisciplinary look at the challenges around the quickly-changing energy landscape. Energy Engineering majors take the same strong quantitative base that other Berkeley engineers take, including physics, linear algebra, single and multivariable calculus, as well as chemistry. It is currently a small major, but every year since its creation in 2012, the major has increased rapidly, some years even doubling in size.

To start off, E93 (Energy Engineering Seminar) is a great opportunity to learn about energy-related topics. It is a guest speaker seminar targeted at beginners that brings in various people who work in the field of Energy Engineering to give presentations about the main ideas in Energy Engineering.

In addition to these core quantitative classes, a huge variety of other perspectives shed light on the energy issue. Required classes come from departments all over the College of Engineering, as well as outside it. The required Civil Engineering classes generally take a bird’s-eye view, focusing on fluid and energy flows at large scales, as well as the associated chemical reactions. From Mechanical Engineering, EnE pulls in a more rigorous mathematical approach to the physical and energetic interactions between and within systems. Electrical Engineering classes seem to split the difference, with EE 137 A + B (Electric Power Systems) taking a high-level view of power systems and EE 134 (Photovoltaics) taking a detail-oriented view of semiconductors. Statistics, sustainability and economics courses also inform the study of energy. Each of these categories pulls a class from a set of possible classes listed on the degree worksheet. For example Stat 134, IEOR 172, or Math 55 can be used for the statistics requirement. Each brings a different perspective to the intersection of statistics and energy. All of these skills come together in a senior capstone project E 194, which allows students to synthesize and apply the knowledge they have gained to a current problem in energy engineering.

Given the breadth of the Energy Industry, the Energy Engineering curriculum allows for students to choose from a multitude of classes and forge their own path here at Cal. This allows students to obtain an understanding of the industry as a whole while also giving them the opportunity to specialize in their given field of interest. Certain classes are offered during only Fall or only Spring, so early planning is beneficial.
If you are having trouble deciding what classes to take, SES and the Engineering Science department have created a few potential schedules to help guide you! :)
 
Student groups are very important in determining what you want to do with your major, as well as for meeting people with similar interests. (For example, try SES!)
Berkeley Energy \& Resources Collaborative (BERCU) attracts a variety of majors and puts on some really cool events, occasionally fueled by free pizza. Engineering project clubs such as Human Powered Vehicle and CalSol are also great ways to hone skills and meet people.  Also take a look at Engineers for a Sustainable World, a club dedicated to hands-on sustainability projects.  It's very popular both for Energy Engineers and your peers in Environmental Engineering Science!
 
EnE’s interdisciplinary nature allows the pursuit of a variety of different career paths including graduate school, mechanical engineering, solar photovoltaics at either the system level or the semiconductor level, and more. The Career Center sends out newsletters about research or internship opportunities, and signing up for BOTH engineering and environmental newsletters will allow you to see a large range of possibilities. (For more information, see {\fontfamily{qcr}\selectfont career.berkeley.edu/MailList/MailList}.) In addition, SES and the College of Engineering are great resources for whatever career path you are on.
 
Examples of relevant challenges include improving energy efficiency in buildings, transportation, and industrial sectors; reducing emissions of carbon dioxide and other environmental impacts associated with fuel combustion in power plants and engines; designing and managing smart energy grids; developing non-fossil energy systems including wind, solar, geothermal, and nuclear power; and developing and improving production, storage, and distribution infrastructure needed for electric vehicles, hydrogen fuel cells, and biofuels.

\textbf{Final Note}: Many EnEs recommend taking Thermodynamics early (to account for scheduling), and to avoid taking both CE 107 and ERG 100 (the content is quite similar). For a list of FAQs about the major and other information, visit {\fontfamily{qcr}\selectfont \seqsplit{engineeringscience.berkeley.edu/energy-engineering}}.