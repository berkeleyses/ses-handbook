\chapter*{Computing \& Scheduling Resources}
\addcontentsline{toc}{chapter}{Computing \& Scheduling Resources}

Internet access and email are integral to your educational experience. Most courses have web sites for class information, and professors often use email to communicate with their students. It is highly recommended that you obtain your own computer. But if this is not possible, don’t worry, there are plenty of options.

On campus, there are several facilities (including libraries) with computers and printers that are available free of charge. Some libraries (including Moffitt) have computers for short-term checkout. If you are enrolled in an EECS class, you will also have access to the Cory and Soda workstations, including 200 pages of free printing per semester.

Open Computing Facility (OCF) accounts are available to all students free of charge. The lab (located in Hearst Gym) offers a limited amount of free printing, web hosting, and other services. See more at {\fontfamily{qcr}\selectfont ocf.berkeley.edu}. (Fun fact: OCF hosts our website.)

Software can be downloaded from {\fontfamily{qcr}\selectfont software.berkeley.edu} for academic and personal use. This includes Adobe Creative Cloud Suite (including Illustrator and Photoshop), Microsoft Office 365 (including Word 2016), Windows 10, Mathematica, and MATLAB. If you’re working with data, be sure to check out computing tools at the Berkeley Institute for Data Science ({\fontfamily{qcr}\selectfont bids.berkeley.edu}) and the D-Lab ({\fontfamily{qcr}\selectfont dlab.berkeley.edu}).

Scheduling is always a nightmare, but there are important resources to help you out. Most important are CalCentral ({\fontfamily{qcr}\selectfont calcentral.berkeley.edu}), which assists you with all of your scheduling needs; and NinjaCourses ({\fontfamily{qcr}\selectfont ninjacourses.com}), which has course ratings submitted by other students so you can easily compare professors from the student perspective. UC Berkeley stopped updating this site, but the reviews of professors are still helpful.  Some people also use ({\fontfamily{qcr}\selectfont ratemyprofessors.com}) to find opinions on classes and lecturers, but be careful, since most people who post on that site are heavily biased either for or against the professors.
{\fontfamily{qcr}\selectfont classes.berkeley.edu} is also a good resource that can help with scheduling. An unofficial but very useful site that tracks enrollment numbers and waitlist statistics is {\fontfamily{qcr} \selectfont berkeleytime.com}
Remember that you can always talk to your peers (including SES members), your ESS Adviser, your Departmental Adviser, or your Faculty Adviser for in-depth information about courses, research, internships, and other academic planning.


