\chapter*{Graduate School}
\addcontentsline{toc}{chapter}{Graduate School}

\section*{How to Find a School}

Start by meeting with your professor and/or faculty adviser during office hours to talk about suitable schools and graduate programs aligned with your interests.

To find a description of universities and their graduate programs, you may want Peterson’s Guide to Graduate Schools. Another useful resource is The Gourman Report, which is a compilation of “Top 25” lists of graduate schools for many specific programs, including bioengineering and environmental engineering. Also, ask your professors or faculty adviser about which programs have good reputations in your particular field. Once you have a tentative list of schools, you could go to the Career and Graduate School Services on 2440 Bancroft Way and look at their catalogs. During the summer before your senior year, call or email the graduate schools for applications and a description of their graduate program. Most schools will first send you a catalog and an application later. If you plan to enter graduate school in the fall, application deadlines usually range from December to February. Graduate schools have either a bioengineering or a biomedical engineering graduate program depending on how closely associated the program is to its medical school. Most programs are for a Ph.D. degree and to a lesser extent the MS degree. Some schools also have MD/Ph.D. programs for the biomedical sciences/engineering that may also be worth looking into.

\section*{Letters of Recommendation}

Most graduate school applications require at least three letters of recommendation. Students usually get letters of recommendations from professors, graduate students, or employers. Graduate schools are not only looking at your academic ability, but also at your ability to interact and work with other people. If you have worked for any professors or have had an internship in industry, make sure you ask your professor or supervisor to write you a letter of recommendation.

\section*{Graduate Record Examinations (GRE)}

The GRE is a standardized test, the scores of which are accepted by most graduate schools across the US and also in many other countries as one of the criteria for consideration for admissions to graduate programs. Among the graduate programs that accept your scores in the GRE, engineering programs are the most lucrative. A combination of high scores in GRE, engineering degrees, and lots of hard work is the key to a successful career in the field of engineering!

Although most graduate schools have worked out various criteria for selecting the applicants who can be granted admissions, your performance in the GRE is one criterion that will be considered by most graduate schools. The weight given to your GRE scores may vary from school to school, but it is a well-known fact that competitive or good GRE scores will definitely tilt the scales in your favor if you are being compared with other applicants. While considering your scores in GRE, engineering colleges or graduate schools offering majors in engineering pay more emphasis on your Quantitative Reasoning score, which is the math score. As it is necessary for an engineering aspirant to be proficient in math, your Quantitative Reasoning score will provide an insight into your skill levels in math. Although your scores in the other two GRE sections are also important, you should aim for scoring very high in the Quantitative Reasoning test section of the GRE if you wish to graduate with a major in engineering subjects.

\section*{Medical School}

You \textbf{can} apply for medical school even if you stick with Engineering Science.

Review medical school requirements \textbf{now} to see what classes you need to take before you apply. Remember that AP units are generally not accepted as credit for lower division classes.  You must take the MCAT approximately 12-18 months before your expected entrance to medical school. Many helpful books and handouts can be found at Career and Graduate School Services on 2440 Bancroft Way. There are also several societies on campus that are support networks for pre-medical students. More information on the MCAT: {\fontfamily{qcr}\selectfont www.aamc.org/students/services/343550/mcat2015.html} 

