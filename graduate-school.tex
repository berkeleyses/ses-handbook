\chapter*{Graduate School}
\addcontentsline{toc}{chapter}{Graduate School}

\section*{How to Find a School}

Start by meeting with your professor and/or faculty adviser during office hours to talk about suitable schools and graduate programs aligned with your interests. Ask your professors or faculty adviser about which programs have good reputations in your particular field. If you are interested in research, you should also browse the lab pages associated with different university labs -- every school has its own specialty! 

Many programs in the sciences (e.g. Physics, Chemistry, Biology) typically want Ph.D. students, while engineering programs tend to offer more MS degrees. If you are not sure whether you would want to pursue research, you should actively look for research internships in the summers leading up to your senior year! Amassing experience in different labs and work situations is a great way to build a convincing application and to meet professors who would be willing to write you a letter of recommendation!

During the summer before your senior year, compile a list of schools (and for Ph.D's, a list of professors at each school whose work you are interested in). Email the graduate schools for information about the application process and email professors letting them know you are interested in their school. If you plan to enter graduate school in the fall, application deadlines usually range from December to February. 
\section*{Letters of Recommendation}

Most graduate school applications require at least three letters of recommendation. Students usually get letters of recommendations from professors, graduate students, or employers. Graduate schools are not only looking at your academic ability, but also at your ability to interact and work with other people. If you have worked for any professors or have had an internship in industry, make sure you ask your professor or supervisor to write you a letter of recommendation. As much as possible, you want your professor to be able to list \emph{specific} details about you, your performance, and why they think you are a good student -- so make sure to keep your advisor and employers updated about what you do!

\section*{Graduate Record Examinations (GRE)}

The GRE is a standardized test, the scores of which are accepted by most graduate schools across the US and also in many other countries as one of the criteria for consideration for admissions to graduate programs. Among the graduate programs that accept your scores in the GRE, engineering programs are the most lucrative. A combination of high scores in GRE, engineering degrees, and lots of hard work is the key to a successful career in the field of engineering!

Although most graduate schools have worked out various criteria for selecting the applicants who can be granted admissions, your performance in the GRE is one criterion that will be considered by most graduate schools. The weight given to your GRE scores may vary from school to school, but it is a well-known fact that competitive or good GRE scores will definitely tilt the scales in your favor if you are being compared with other applicants. It's not necessarily true the great GRE scores guarantee admission to a great graduate school, but doing poorly can significantly hurt your chances. While considering your scores in GRE, engineering colleges or graduate schools offering majors in engineering pay more emphasis on your Quantitative Reasoning score, which is the math score. Although your scores in the other two GRE sections are also important, you should aim for scoring very high in the Quantitative Reasoning test section of the GRE if you wish to graduate with a major in engineering subjects.


